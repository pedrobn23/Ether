\documentclass[11pt,a4paper]{article}

% Packages
\usepackage[utf8]{inputenc}
\usepackage[spanish, es-tabla]{babel}
\usepackage{caption}
\usepackage{listings}
\usepackage{adjustbox}
\usepackage{enumitem}
\usepackage{boldline}
\usepackage{amssymb, amsmath}
\usepackage[margin=1in]{geometry}
\usepackage{xcolor}
\usepackage{soul}

% Meta
\title{Título}
\author{Pedro Bonilla Nadal}
\date{\today}

% Custom
\providecommand{\abs}[1]{\lvert#1\rvert}
\setlength\parindent{0pt}
\definecolor{Light}{gray}{.90}
\newcommand\ddfrac[2]{\frac{\displaystyle #1}{\displaystyle #2}}

\begin{document}
\maketitle

Con internet extendido por todo el mundo, la trasmisión de información se ha convertido en una actividad disponible a toda la población. En la actulidad, movimientos  bajados en las nuevas tecnología, mediante el poder de las cosas hechas por defecto, protocolos consensuados y respeto voluntario al contrato social han hecho posible el uso de internet para crear un sistema de trasferencia de valor descentralizado, es decir, una moneda (o criptomoneda) que no necesite el respaldo de un banco que central que la acuñe, gestione y proteja. El sistema está expandido alrededor del mundo y el virtualmente de uso libre. El ejemplo más famoso de estos sistemas es el conocido Bitcoin. Estos sistemas pueden verse como una versión especializada de sistemas de estados basados en transación criptográficamente segura.\\

Ethereum es un proyecto que aspira a  construir la generalizacíon de esta tecnología, en la que todos ls sistemas de estados basados en transacción pueden ser desarrollados. Además, pretende ofrecer al desarrollador un framework confiable para la tramisión de mensajes, con ecriptación E2EE.

\end{document}
