% Tipo de documento (presentación)
\documentclass[usenames,dvipsnames]{beamer}

% Cargar el tema
\usetheme{metropolis}

% Configuración de LaTeX
\usepackage[spanish]{babel}
\usepackage[utf8]{inputenc}

% Configuración básica del tema
\metroset{
  % tema oscuro ('dark') o claro ('light'). No tiene efecto al usar la
  % paleta de colores más adelante
  background=light,
  % 'none' para eliminar la diapositiva inicial de cada sección
  sectionpage=progressbar,
  % 'progressbar' o 'simple' para añadir una diapositiva inicial a cada subsección
  subsectionpage=none,
  % contador de página: 'none', 'counter' o 'fraction'
  numbering=none,
  % barra de progreso: 'none', 'head', 'frametitle' o 'foot'
  progressbar=frametitle,
  % fondo de los bloques estilo teorema: 'transparent' o 'fill'
  block=fill,
}


% Paleta de colores
\definecolor{accent}{RGB}{151, 186, 88}
\colorlet{darkaccent}{accent!70!black}
\definecolor{foreground}{RGB}{0, 0, 0}
\definecolor{background}{RGB}{255, 255, 255}

% Insertar los colores en el tema
\setbeamercolor{normal text}{fg=foreground, bg=background}
\setbeamercolor{alerted text}{fg=darkaccent, bg=background}
\setbeamercolor{example text}{fg=foreground, bg=background}
\setbeamercolor{frametitle}{fg=background, bg=accent}

\setbeamercolor{headtitle}{fg=background!70!accent,bg=accent!90!foreground}
\setbeamercolor{headnav}{fg=background,bg=accent!90!foreground}
\setbeamercolor{section in head/foot}{fg=background,bg=accent}

%
\defbeamertemplate*{headline}{miniframes theme no subsection}{
  % Caja para mostrar título y autor encima de cada diapositiva
  % Nosotros no 
  %% \begin{beamercolorbox}[ht=2.5ex,dp=1.125ex,
  %%     leftskip=.3cm,rightskip=.3cm plus1fil]{headtitle}
  %%   {\usebeamerfont{title in head/foot}\insertshorttitle}
  %%   \hfill
  %%   \leavevmode{\usebeamerfont{author in head/foot}\insertshortauthor}
  %% \end{beamercolorbox}
  %% \begin{beamercolorbox}[colsep=1.5pt]{upper separation line head}
  %% \end{beamercolorbox}

  % Caja para mostrar navegación encima de cada diapositiva
  \begin{beamercolorbox}{headnav}
    \vskip2pt\insertnavigation{\paperwidth}\vskip2pt
  \end{beamercolorbox}
  \begin{beamercolorbox}[colsep=1.5pt]{lower separation line head}
  \end{beamercolorbox}
}
\title{Ethereum}
\subtitle{Una nueva visión de la cadena de bloques}
\date{\today}
\institute{Universidad de Granada}
\author{Ana Peña\\Pedro Bonilla}
\titlegraphic{\hfill\includegraphics[height=1.5cm]{./images/logo.png}}


\begin{document}
\maketitle
\begin{frame}{Contenidos.}
  \setbeamertemplate{section in toc}[sections numbered]
  \tableofcontents [hideallsubsections]
\end{frame}

\section{ Introducción. }
\section{ Ethereum. }
\subsection{ Historia. }
\subsubsection{ Hardfork y el cisma de ethereum. }
\subsection{ Funcionamiento }

\subsubsection{ Bitcoin como sistema de transición de estados. }
\subsubsection{ Cadena de Bloques de Ethereum. }
\subsubsection{ Cuentas. }
\subsubsection{ Máquina Virtual de Ethereum . }
\subsubsection{ Mensajes, Transacciones y estado de transición. }
\subsubsection{ Contratos inteligentes. }
\subsubsection{ Minería y Prueba de Trabajo. }
\subsubsection{ Hardfork y Sotfork. }
\subsubsection{ Hardfork. }
\subsubsection{ Softfork. }

\section { Aplicaciones. }
\subsection{ Sistemas de Token. }
\subsection{ Sistema de identidad. }
\subsection{ Decentralized File Storage. }
\subsection{ Organizaciones descentralizadas. }

\section{ Particularidades. }
\subsection{ Ether. }
\subsection{ DoS attack. }
\subsection{ Escalabilidad. }
\subsubsection{ Ataque 51 \%. }
\subsection{ Minería centralizada .}

\section{Conclusiones}
\subsection{Conclusiones}
\begin{frame}[standout]
  ¡Muchas gracias!
\end{frame}
\end{document}

%\seccion{estructura.tex}  % Estructura
%\seccion{aspecto.tex}     % Aspecto
%\seccion{otros.tex}       % Otros objetos
%\seccion{enlaces.tex}     % Enlaces de interés

\end{document}
