


\textbf{¿CÓMO FUNCIONA ETHEREUM?\\}


Una vez sabemos lo que es ethereum, profundizamos en el funcionamiento de la plataforma.
Al usar ethereum, la ‘app’ no requiere ninguna entidad para almacenar y controlar sus datos. Para conseguirlo, ethereum hace uso del protocolo de bitcoin y su diseño de la cadena de bloques,aunque lo ajusta de manera que puede respaldar aplicaciones además del dinero.
Sin embargo, ethereum persigue abstraerse del diseño de bitcoin con el fin de que los
desaprobadores puedan crear aplicaciones o acuerdos que incluyan medidas adicionales, nuevas
normas de propiedad, formatos alternativos de transacciones o diferentes maneras de transferir
estados(?).
El objetivo del lenguaje de programación ‘Turing-completo’ de ethereum es permitir a los
desarrolladores escribir más programas en los que las transacciones en la cadena de bloques
puedan gobernar y automatizar resultados específicos. Esta flexibilidad que ofrece ethereum es
quizás su innovación principal.
Cadena de Bloques de Ethereum
La estructura de la cadena de bloques de ethereum es muy similar a la de bitcoin, dado que se
trata de un registro compartido de la historia de transacciones completa. Cada nodo en la red
almacena una copia de este historial.
La gran diferencia con ethereum es que sus nodos almacenan también el estado más reciente de
cada contrato inteligente, además de todas las transacciones de ether.
Para cada aplicación de ethereum, la red tiene que mantener un seguimiento del ‘estado’ o
información actual de todas estas aplicaciones, incluyendo el saldo de cada usuario, todo el
código del contrato inteligente y dónde se almacena todo.
Bitcoin usa salidas de transacción no utilizadas para rastrear quién tiene cuánto bitcoin.
Aunque suena complejo, la idea es bastante simple. Cada vez que se hace una transacción de
bitcoin, la red ‘rompe’ la cantidad total como si fuera dinero impreso, emitiendo bitcoins de vuelta
de una forma que hace que la información manejada se comporte como las monedas o cambio
físico.
Para efectuar futuras transacciones, la red de bitcoin tiene que añadir todas las piezas de cambio,
que se clasifican en ‘gastadas’ o ‘no gastadas’.
Ethereum, por otro lado, utiliza cuentas.
Como en fondos de cuentas bancarias, las ‘fichas’ de ether aparecen en una cartera, y pueden ser
portados (por así decirlo) a otra cuenta. Los fondos siempre están en algún sitio, pero no tienen lo
que podría llamarse una relación continua.Máquina Virtual de Ethereum
Con ethereum, cada vez que se usa un programa, una red de miles de computadores lo procesa.
Los contratos escritos en un lenguaje de programación específico de contrato inteligente se
compilan en ‘bytecode’, lo que una prestación llamada ‘ethereum virtual machine’ (EVM) puede
leer y ejecutar.
Todos los nodos ejecutan este contrato usando sus EVMs.
Cada vez que un usuario realiza alguna acción, todos los nodos de la red tienen que estar de
acuerdo en que ese cambio se ha efectuado.
El objetivo aquí es que la red de mineros y nodos tomen la responsabilidad de transferir el cambio
de estado a estado, en lugar de cualquier autoridad como PayPal o un banco. Los mineros de
bitcoin validan el cambio de propiedad de bitcoins de una persona a otra. La EVM ejecuta un
contrato con las reglas que el desarrollador programó inicialmente.
El cálculo real en la EVM se consigue mediante un lenguaje ‘bytecode’ basado en pilas, pero los
desarrolladores pueden escribir contratos inteligentes en lenguajes de alto nivel como Solidity o
Serpent, más fáciles de leer y escribir para las personas.
Los mineros son los que evitan un mal comportamiento. Por ejemplo, deben asegurarse de que
nadie gasta dinero más de una vez, y rechazar contratos inteligentes que no se han pagado.
Existen varios miles de nodos de ethereum ahí fuera, y cada uno de ellos está compilando y
ejecutando el mismo código.
Podemos pensar que todo esto tiene un coste mucho mayor al de cálculos ordinarios, lo cual es
cierto. Por ello la red solo debe usarse para casos de uso particular.
El tutorial oficial de desarrollo de ethereum reconoce esta ineficacia, declarando:
"A grandes rasgos, una buena heurística para usar es que no podrás hacer nada en la EVM que
no puedas hacer en un teléfono inteligente de 1999".

